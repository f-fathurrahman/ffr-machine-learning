\documentclass[bahasa,11pt,aspectratio=169]{beamer}

\usepackage{amsmath}
\usepackage{amssymb}

\usefonttheme[onlymath]{serif}

\setlength{\parskip}{\smallskipamount}
\setlength{\parindent}{0pt}

\setbeamersize{text margin left=5pt, text margin right=5pt}

\usepackage{amsmath}
\usepackage{amssymb}
\usepackage{braket}

\usepackage{minted}
\newminted{python}{breaklines,fontsize=\normalsize,texcomments=true}
\newminted{bash}{breaklines,fontsize=\normalsize,texcomments=true}
\newminted{text}{breaklines,fontsize=\normalsize,texcomments=true}

\newcommand{\txtinline}[1]{\mintinline[fontsize=\normalsize]{text}{#1}}
\newcommand{\pyinline}[1]{\mintinline[fontsize=\normalsize]{python}{#1}}

\definecolor{mintedbg}{rgb}{0.95,0.95,0.95}
\usepackage{mdframed}

\BeforeBeginEnvironment{minted}{\begin{mdframed}[backgroundcolor=mintedbg]}
\AfterEndEnvironment{minted}{\end{mdframed}}

\setcounter{secnumdepth}{3}
\setcounter{tocdepth}{3}

\makeatletter

 \newcommand\makebeamertitle{\frame{\maketitle}}%
 % (ERT) argument for the TOC
 \AtBeginDocument{%
   \let\origtableofcontents=\tableofcontents
   \def\tableofcontents{\@ifnextchar[{\origtableofcontents}{\gobbletableofcontents}}
   \def\gobbletableofcontents#1{\origtableofcontents}
 }

\makeatother

\usepackage{babel}

\begin{document}


\title{Pengenalan Principal Component Analysis}
%\author{Estiyanti Ekawati \\
%Fadjar Fathurrahman}
\institute{
Teknik Fisika \\
Institut Teknologi Bandung
}
\date{} % 16 Oktober 2019

\frame{\titlepage}


\begin{frame}

Bahan-bahan berikut ini diambil dari:

Simon Rogers and Mark Girolami. A First Course in Machine Learning. 2nd Edition.
CRC Press. 2017.

\end{frame}

%=========================
\begin{frame}
\frametitle{Proyeksi data}

\begin{itemize}
\item Misalkan kita memiliki dataset yang terdiri dari $N$ objek, $y_{n}$, $n=1,2,\ldots,N$.
%
\item Setiap objek adalah vektor dengan dimensi $M$ ($M$ dapat berupa jumlah fitur dari
data yang kita miliki).
%
\item Sebagian besar model mesin pembelajar yang memiliki parameter yang akan
bertambah banyak jika dimensi dari data semakin banyak.
%
\item Data dengan dimensionalitas yang tinggi juga sulit untuk divisualisasi.
%
\item Terkadang transformasi data $M$-dimensi ke representasi $D$ dimensi (dengan $D$ < $M$)
diperlukan. Proses ini dikenal dengan nama proyeksi.
%
\item Idealnya proyeksi ini tetap memiliki properti menarik dari data yang kita
ingin pelajari.
\end{itemize}

\end{frame}




\begin{frame}[fragile]
\frametitle{Contoh proyeksi}

\begin{center}
\includegraphics[scale=1.0]{images_priv/tangan.pdf}
\end{center}

Objek tangan (3d) diproyeksikan ke 2d (bayangan).

\end{frame}


\begin{frame}[fragile]
\frametitle{Contoh proyeksi}

\begin{center}
\includegraphics[scale=1.0]{images_priv/gambar_7_1_c.pdf}
\end{center}

Proyeksi suatu data ke sumbu $x$.

\end{frame}


\begin{frame}
\frametitle{Bagaimana cara mengukur derajat "menarik/penting" dari data?}

\begin{center}
\includegraphics[scale=1.0]{images_priv/gambar_7_2_a.pdf}
\end{center}

Contoh proyeksi data ke dua arah (A dan B).

\end{frame}



\begin{frame}
\frametitle{Principal Component Analysis (PCA)}

\begin{itemize}
\item PCA adalah metode yang sering digunakan untuk melakukan proyeksi data ke dimensi
yang lebih rendah.
\item PCA adalah proyeksi linear: setiap dimensi proyeksi adalah kombinari linear dari
dimensi asli. Jika kita melakukan proyeksi dari $M$ ke $D$ dimensi, PCA akan mendefisikan
$D$ vektor, $\mathbf{w}_{d}$, yang masing-masingnya berdimensi $M$. Elemen ke-$d$ dari
proyeksi $x_{nd}$ (di mana $[x_{n} = x_{n1},\ldots,x_{nD}]^{\mathsf{T}}$ ) adalah:
$$
x_{nd} = \mathbf{w}^{\mathsf{T}}_{d} \mathbf{y}_{n}
$$
\end{itemize}

\end{frame}


\begin{frame}
\frametitle{PCA}

\begin{itemize}
\item PCA menggunakan variansi pada ruang proyeksi sebagai kriteria untuk memilih
$\mathbf{w}_{d}$.
%
\item Misalnya: $\mathbf{w}_{1}$ adalah proyeksi yang akan membuat variansi pada $x_{n1}$ semaksimal mungkin.
%
\item Dimensi proyeksi kedua juga dipilih untuk memaksimalkan variansi, namum $\mathbf{w}_{2}$
harus ortogonal terhadap $\mathbf{w}_{1}$:
$$
\mathbf{w}_{2}^{\mathrm{T}} \mathbf{w}_{1} = 0
$$
%
\item Begitu juga untuk dimensi proyeksi yang ketiga dan seterusnya. Secara umum:
$$
\mathbf{w}_{i}^{\mathrm{T}} \mathbf{w}_{j} = 0 \, \forall\, j \neq i
$$
\item PCA juga menambahkan konstrain bahwa tiap $\mathbf{w}_{i}$ memiliki panjang 1.
$$
\mathbf{w}_{i}^{\mathsf{T}} \mathbf{w}_{i} = 1
$$

\end{itemize}

\end{frame}


\begin{frame}
\frametitle{Prosedur PCA}

Dapat ditunjukkan bahwa $\mathbf{w}_{i}$ dapat diperoleh dari persamaan eigen:
$$
\mathbf{C} \mathbf{w} = \lambda \mathbf{w}
$$
dengan $\mathbf{C}$ adalah matriks kovariansi:
$$
\mathbf{C} = \frac{1}{N} \sum_{n=1}^{N}
(\mathbf{y}_{n} - \bar{\mathbf{y}})
(\mathbf{y}_{n} - \bar{\mathbf{y}})^{\mathsf{T}}
$$
atau:
$$
\mathbf{C} = \frac{1}{N} \mathbf{Y}^{\mathsf{T}} \mathbf{Y}
$$
dan nilai eigen $\lambda$ adalah variansi dari data yang diprojeksikan ke arah $\mathbf{w}$.

\end{frame}

\begin{frame}
\frametitle{Prosedur PCA}

\begin{itemize}
\item Transformasi data sehingga memiliki rata-rata nol dengan cara mengurangi setiap
titik data dengan rata-rata sampel:
$$
\bar{\mathbf{y}} = \frac{1}{N} \sum_{n=1}^{N} \mathbf{y}_{n}
$$
\item Hitung matriks kovariansi.
\item Cari pasangan nilai eigen dari matriks kovariansi.
\item Cari eigenvektor dengan $D$ nilai eigen tertinggi.
\item Buat proyeksi data:
$$
\mathbf{X} = \mathbf{Y}\mathbf{W}
$$
di mana $\mathbf{W}$ adalah matriks $M \times D$ yang dibuat dari $D$ vektor eigen
dari matriks kovariansi.
\end{itemize}

\end{frame}


\begin{frame}[fragile]
\frametitle{Tugas: PCA pada data sintetik}

\begin{itemize}
\item Buat data sintetik yang memiliki struktur klaster. Misalnya
\begin{pythoncode}
Y_1 = np.random.randn(20,2) # jumlah data adalah 20.
Y_2 = np.random.randn(20,2) + 5.0
Y_3 = np.random.randn(20,2) - 5.0
Y = np.concatenate( (Y_1, Y_2, Y_3), axis=0 )
\end{pythoncode}

\item Tambahkan beberapa dimensi randam pada data yang tidak memiliki struktur
klaster.
\begin{pythoncode}
Ndata = Y.shape[0]
Y = np.concatenate( (Y, np.random.randn(Ndata,5)), axis=1) # tambah data 5 dimensi
\end{pythoncode}
\item Aplikasikan prosedur PCA pada data tersebut. Plot data yang sudah tereduksi
dimensionalitasnya.
\end{itemize}

Bandingkan hasil yang Anda peroleh dengan menggunakan pustaka Scikit Learn.
Apakah ada perbedaan yang Anda amati? Jelaskan apa yang mungkin menyebabkan
perbedaan tersebut jika ada.

\end{frame}


\begin{frame}
\frametitle{Hint}
\begin{itemize}
\item Lakukan visualisasi pada data sintetik yang dibuat. Buat scatterplot
untuk pasangan data pada tiap dimensi (fitur).
\item Untuk menghitung nilai dan vektor eigen: \texttt{np.linalg.eig}
\item Untuk melakukan perkalian matriks: \texttt{np.matmul} (jika tipe data adalah
\texttt{ndarray}).
\end{itemize}
\end{frame}


\end{document}
