\documentclass[english,10pt,aspectratio=169,fleqn]{beamer}

\usepackage{amsmath} % load this before unicode-math
\usepackage{amssymb}
\usepackage{mathabx}
%\usepackage{unicode-math}

\usepackage{fontspec}
\setmonofont{DejaVu Sans Mono}
%\setmathfont{STIXMath}
%\setmathfont{TeX Gyre Termes Math}

\usefonttheme[onlymath]{serif}

\setlength{\parskip}{\smallskipamount}
\setlength{\parindent}{0pt}

%\setbeamersize{text margin left=5pt, text margin right=5pt}

\usepackage{amsmath}
\usepackage{amssymb}
\usepackage{braket}

\usepackage{minted}
\newminted{julia}{breaklines,fontsize=\scriptsize,texcomments=true}
\newminted{python}{breaklines,fontsize=\scriptsize,texcomments=true}
\newminted{bash}{breaklines,fontsize=\scriptsize,texcomments=true}
\newminted{text}{breaklines,fontsize=\scriptsize,texcomments=true}

\newcommand{\txtinline}[1]{\mintinline[fontsize=\scriptsize]{text}{#1}}
\newcommand{\jlinline}[1]{\mintinline[fontsize=\scriptsize]{julia}{#1}}

\definecolor{mintedbg}{rgb}{0.95,0.95,0.95}
\usepackage{mdframed}

%\BeforeBeginEnvironment{minted}{\begin{mdframed}[backgroundcolor=mintedbg]}
%\AfterEndEnvironment{minted}{\end{mdframed}}

\setcounter{secnumdepth}{3}
\setcounter{tocdepth}{3}

\makeatletter

 \newcommand\makebeamertitle{\frame{\maketitle}}%
 % (ERT) argument for the TOC
 \AtBeginDocument{%
   \let\origtableofcontents=\tableofcontents
   \def\tableofcontents{\@ifnextchar[{\origtableofcontents}{\gobbletableofcontents}}
   \def\gobbletableofcontents#1{\origtableofcontents}
 }

\makeatother

\usepackage{babel}

\begin{document}


\title{Bayesian Inference (Overview to Homework)}
\subtitle{TF4063}
\author{Fadjar Fathurrahman}
\institute{
Program Studi Teknik Fisika\\
Institut Teknologi Bandung
}
\date{}


\frame{\titlepage}


\begin{frame} % ---------------------------------------------------------------
\frametitle{An example of nonconjugate models: binary response}

\begin{columns}

\begin{column}{0.5\textwidth}
{\centering
\includegraphics[scale=1.0]{images_priv/Rogers_Fig_4_1.pdf}
\par}
\end{column}

\begin{column}{0.5\textwidth}
\begin{itemize}
\item Data has two attributes: $x_1$ and $x_2$.
\item Target only has two possible values: circle (0) or square (1).
\item This is a binary classification problem.
\end{itemize}
\end{column}

\end{columns}

\end{frame} % -----------------------------------------------------------------



\begin{frame} % ---------------------------------------------------------------
\frametitle{Model}

The following representation will be used for our data
\begin{equation*}
\mathbf{x}_{n} = \begin{bmatrix}
x_{n1} \\
x_{n2}
\end{bmatrix},\,\,
\mathbf{w} = \begin{bmatrix}
w_{1} \\
w_{2}
\end{bmatrix},\,\,
\mathbf{X} = \begin{bmatrix}
x_{1}^{\mathsf{T}} \\
x_{1}^{\mathsf{T}} \\
\vdots \\
x_{N}^{\mathsf{T}}
\end{bmatrix}
\end{equation*}
%
Our model (with parameters $\mathbf{w}$) will allow us to predict $t_{\mathrm{new}}$
for some new observation $\mathbf{x}_{\mathrm{new}}$.

We need to compute the posterior density over the parameters of the model.
According to Bayes' rule, this is calculated as
\begin{equation*}
p(\mathbf{w} | \mathbf{t}, \mathbf{X}) =
\frac{p(\mathbf{t}|\mathbf{X},\mathbf{w}) p(\mathbf{w})}%
{p(\mathbf{t}|\mathbf{X})}
\end{equation*}
%
Marginal likelihood
\begin{equation*}
p(\mathbf{t}|\mathbf{X}) = \int p(\mathbf{t} | \mathbf{X}, \mathbf{w})
p(\mathbf{w})\,\mathrm{d}\mathbf{w}
\end{equation*}

\end{frame}


\begin{frame} % ----------------------------------------------
\frametitle{Prior and likelihood}

We will use Gaussian density for the prior.
\begin{equation*}
p(\mathbf{w} | \sigma^2) = \mathcal{N}(\mathbf{0},\sigma^2\mathbf{I})
\end{equation*}

Assuming that the elements of $\mathbf{t}$ are conditionally independent,
the likelihood can be written as
\begin{equation*}
p(\mathbf{t} | \mathbf{X}, \mathbf{w}) = \prod_{n=1}^{N}
p(t_{n} | \mathbf{x}_{n}, \mathbf{w})
\end{equation*}
where $t_{n}$ are a binary variable indicating the class (0 or 1) of the $n$-th
object.

$t_{n}$ will be modeled using binary random variable, $T_{n}$ which is characterized
by the probability that the class is 1 (the probability of belonging to class 0 is
1 minus the probability of belonging to class 1). Therefore, we can write each of
the $n$ likelihood terms as a probability.
\begin{equation*}
p(\mathbf{t} | \mathbf{X}, \mathbf{w}) = \prod_{n=1}^{N}
P(T_{n}=t_{n} | \mathbf{x}_{n}, \mathbf{w})
\end{equation*}

\end{frame}


\begin{frame}
\frametitle{Likelihood}

Our task now is to choose a function of $\mathbf{x}_{n}$ and $\mathbf{w}$
that produces a probability.

A popular technique is to take a simple linear
function, such as
$f(\mathbf{x}_{n};\mathbf{w}) = \mathbf{w}^{\mathsf{T}}\mathbf{x}_{n}$,
and then pass the result through a second function that squashes its output
to ensure it produces a valid probability.

One example of such squashing function is the \textit{sigmoid} function.
The probability of getting $T_{n}=1$ thus can be written as
\begin{equation*}
P(T_{n}=1 | \mathbf{x}_{n},\mathbf{w}) =
\frac{1}{1 + \exp(-\mathbf{w}^{\mathsf{T}}\mathbf{x}_{n})}
\end{equation*}
The probability of getting $T_{n}=0$ can be computed from
\begin{align*}
P(T_{n}=0 | \mathbf{x}_{n},\mathbf{w}) & = 1 - P(T_{n}=1 | \mathbf{x}_{n},\mathbf{w})
 = 1 - \frac{1}{1 + \exp(-\mathbf{w}^{\mathsf{T}}\mathbf{x}_{n})} \\
& = \frac{\exp(-\mathbf{w}^{\mathsf{T}}\mathbf{x}_{n})}%
{1 + \exp(-\mathbf{w}^{\mathsf{T}}\mathbf{x}_{n})}
\end{align*}

\end{frame}


\begin{frame} % --------------------------------------------------------------
\frametitle{Likelihood}

We can combine the two probabilities to produce a single expression
\begin{equation*}
P(T_{n}=t_{n} | \mathbf{x}_{n}, \mathbf{w}) = 
P(T_{n}=1 | \mathbf{x}_{n}, \mathbf{w})^{t_{n}}
P(T_{n}=0 | \mathbf{x}_{n}, \mathbf{w})^{1-t_{n}}
\end{equation*}
where the observed data $t_{n}$ switches the relevant term on and
the other off.

Substituting this to the expression for the likelihood, we obtain:
\begin{equation*}
P(\mathbf{t} | \mathbf{X}, \mathbf{w}) = \prod_{n=1}^{N}
\left( \frac{1}{1 + \exp(-\mathbf{w}^{\mathsf{T}}\mathbf{x}_{n})} \right)^{t_{n}}
\left( \frac{\exp(-\mathbf{w}^{\mathsf{T}}\mathbf{x}_{n})}%
{1 + \exp(-\mathbf{w}^{\mathsf{T}}\mathbf{x}_{n})} \right)^{1-t_{n}}
\end{equation*}
\end{frame}

\begin{frame}
\frametitle{Posterior}

Once we have the posterior density, $p(\mathbf{w}|\mathbf{X},\mathbf{t},\sigma^2)$,
we can predict the response (class) of the new objects by taking an expectation w.r.t
this density.
\begin{equation*}
P(t_{\mathrm{new}}=1 | \mathbf{x}_{new},\mathbf{X},\mathbf{t}) =
\mathbf{E}_{p(\mathbf{w}|\mathbf{X},\mathbf{t},\sigma^2)}
\left\{
\frac{1}{1 + \exp(-\mathbf{w}^{\mathsf{T}}\mathbf{x}_{n})}
\right\}
\end{equation*}

In practice, this is not straigtforward. The posterior is not of any standard
form. To evaluate the posterior at a particular $\mathbf{w}$, we need to
evaluate both the numerator and denominator.
\begin{equation*}
p(\mathbf{w} | \mathbf{t}, \mathbf{X}) =
\frac{p(\mathbf{t}|\mathbf{X},\mathbf{w}) p(\mathbf{w})}%
{p(\mathbf{t}|\mathbf{X})}
\end{equation*}

The denominator is the problem, as we cannot analytically perform the integration
required to compute the marginal likelihood:
\begin{equation*}
Z^{-1} = p(\mathbf{t},\mathbf{X},\sigma^2) = 
\int p(\mathbf{t}|\mathbf{X},\mathbf{w}) p(\mathbf{w}|\sigma^2)\,\mathrm{d}\mathbf{w}
\end{equation*}

\end{frame}


\begin{frame} % ---------------------------------------------------------------
\frametitle{Posterior}

We have a function $g(\mathbf{w}; \mathbf{X},\mathbf{t},\sigma^2) \equiv
p(\mathbf{t} | \mathbf{X},\mathbf{w}) p(\mathbf{w}|\sigma^2)$
that we know is proportional to the posterior $p(\mathbf{w} | \mathbf{t}, \mathbf{X}, 
\sigma^2) = Z g(\mathbf{w}; \mathbf{X},\mathbf{t},\sigma^2)$
but we do not know the constant of proportionality $Z$.

We are left with three options:
\begin{itemize}
\item Find the single value of $\mathbf{w}$ that corresponds to the highest
value of the posterior.
\item Approximate the posterior with other density that we can compute analytically.
\item Sample directly from the posterior, knowing only
$g(\mathbf{w}; \mathbf{X},\mathbf{t},\sigma^2)$
\end{itemize}

The first option is not very Bayesian – we will have to make predictions for new
objects based on a single value of w and not a density. It is, however, easy to do and
this makes it a popular technique.

The second option leaves us with a density that
is easy to work with (we can choose any density we like) but if the chosen density
is very different from the posterior, our model will not be very reliable.

The final
option allows us to sample from the posterior (and hence get good approximations
to any expectations that we might require) but can be difficult.

\end{frame}

\begin{frame}
\frametitle{Result: 1st option}

Find the maximum of $g(\mathbf{w},\mathbf{X},\mathbf{t})$ or
$\log(g(\mathbf{w},\mathbf{X},\mathbf{t}))$.

{\centering
\includegraphics[scale=1.0]{images_priv/Rogers_Fig_4_4_b.pdf}
\par}

\end{frame}


\begin{frame} % ---------------------------------------------------------------
\frametitle{Result: 2nd option}

Approximate the posterior with a Gaussian.

{\centering
\includegraphics[scale=1.0]{images_priv/Rogers_Fig_4_7_b.pdf}
\par}
\end{frame}


\begin{frame} % ---------------------------------------------------------------
\frametitle{Result: 3rd option}
  
Using Metropolis-Hastings sampling algorithm.
  
{\centering
\includegraphics[scale=1.0]{images_priv/Rogers_Fig_4_12_e.pdf}
\par}

\end{frame}


\begin{frame}
\frametitle{Homework}

\begin{itemize}
\item Create Jupyter Notebook with documentation (narration)
or scripts (Python) with report (in PDF format)
to solve the binary classification problem
described before using the three schemes described in Chap. 4 of Rogers and Girolami.
\item Optional: test your programs using other dataset (for binary classification).
Example:\\
{\footnotesize\url{https://scikit-learn.org/stable/auto_examples/classification/plot_classifier_comparison.html}}
\item Deadline: 21st Oct 2020.
\item The homework can be done individually or collectively (in a group of two).
\end{itemize}

\end{frame}


\end{document}

