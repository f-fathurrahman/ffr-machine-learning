%\documentclass[a4paper,11pt]{article} % print setting
\documentclass[a4paper,11pt]{article} % screen setting

\usepackage[a4paper]{geometry}
\geometry{verbose,tmargin=1.5cm,bmargin=1.5cm,lmargin=1.5cm,rmargin=1.5cm}

\setlength{\parskip}{\smallskipamount}
\setlength{\parindent}{0pt}

%\usepackage{cmbright}
%\renewcommand{\familydefault}{\sfdefault}

%\usepackage{fontspec}
\usepackage[libertine]{newtxmath}
\usepackage[no-math]{fontspec}
\setmainfont{Linux Libertine O}
%\setmonofont{DejaVu Sans Mono}
\setmonofont{JuliaMono-Regular}


\usepackage{hyperref}
\usepackage{url}
\usepackage{xcolor}

\usepackage{amsmath}
\usepackage{amssymb}

\usepackage{graphicx}
\usepackage{float}

\usepackage{minted}

\newminted{julia}{breaklines,fontsize=\footnotesize}
\newminted{python}{breaklines,fontsize=\footnotesize}

\newminted{bash}{breaklines,fontsize=\footnotesize}
\newminted{text}{breaklines,fontsize=\footnotesize}

\newcommand{\txtinline}[1]{\mintinline[breaklines,fontsize=\footnotesize]{text}{#1}}
\newcommand{\jlinline}[1]{\mintinline[breaklines,fontsize=\footnotesize]{julia}{#1}}
\newcommand{\pyinline}[1]{\mintinline[breaklines,fontsize=\footnotesize]{python}{#1}}

\newmintedfile[juliafile]{julia}{breaklines,fontsize=\footnotesize}
\newmintedfile[pythonfile]{python}{breaklines,fontsize=\footnotesize}

\definecolor{mintedbg}{rgb}{0.90,0.90,0.90}
\usepackage{mdframed}
\BeforeBeginEnvironment{minted}{
    \begin{mdframed}[backgroundcolor=mintedbg,%
        topline=false,bottomline=false,%
        leftline=false,rightline=false]
}
\AfterEndEnvironment{minted}{\end{mdframed}}


\usepackage{setspace}

\onehalfspacing

\usepackage{appendix}


\newcommand{\highlighteq}[1]{\colorbox{blue!25}{$\displaystyle#1$}}
\newcommand{\highlight}[1]{\colorbox{red!25}{#1}}



\begin{document}


\title{Linear Modeling\\
TF4xxx}
\author{Fadjar Fathurrahman}
\date{}
\maketitle

\section{Least Square Approach}

\subsection{Loss function}

Given pair of data $(x,t)$ where $x$ are inputs dan $t$
are targets, a linear model with parameter $(w_0, w_1)$ can
be written as:
\begin{equation}
t = f(x; w_0, w_1) = w_0 + w_1 x
\label{eq:model_linear_01}
\end{equation}

We are now left with the task of choosing the best parameters $(w_0, w_1)$
fir this model.
We need to quantify how good the model is.
One metric that we can use to quantify this is the squared difference (or
error) between target and model prediction. For $n$-th data we can write
$$
\mathcal{L}_n \equiv \left( t_n - f(x_n; w_0, w_1) \right)^2
$$
%
By summing contributions from all data:
\begin{equation}
\mathcal{L} = \frac{1}{N} \sum_{n=1}^{N} \mathcal{L}_n
\label{eq:loss_function_01}
\end{equation}
We will call this quantity as \highlight{loss function}
and we want to this quantity to be as small as possible.

\subsection{Minimizing loss function}

We can find the parameters $(w_{0},w_{1})$ by using minimization procedures:
\begin{equation}
\arg\min_{w_{0},w_{1}} \frac{1}{N} \sum_{n=1}^{N} \mathcal{L}_{n}
\end{equation}

For our particular case of Eq. \eqref{eq:loss_function_01}, we can found this analitically,
i.e. calculating the first derivatives of $\mathcal{L}$ with respect to $w_0$ and $w_1$, equating
them to zero, and solve the resulting equations for $w_0$ and $w_1$.
For more general cases, we can use various numerical optimization procedures such as
gradient descent.

We begin by writing our loss function as:
\begin{align*}
\mathcal{L} & = \frac{1}{N} \sum_{n=1}^{N} \left( t_n - (w_0 + w_1 x_{n}) \right)^2 \\
& = \frac{1}{N} \sum_{n=1}^{N} \left( w_1^2 x_n^2 + 2w_{1}x_{n}(w_0 - t_n) + w_0^2 - 2w_0 t_n + t_n^2 \right)
\end{align*}
%
Now we find the first derivatives of $\mathcal{L}$ with respect to
$w_0$, $w_1$ and equating them to zero.
\begin{equation*}
\frac{\partial\mathcal{L}}{\partial w_1} = 2w_1 \frac{1}{N} \left( \sum_{n=1}^{N} x_n^2 \right) +
\frac{2}{N} \left( \sum_{n=1}^{N} x_{n} (w_0 - t_n) \right)
\end{equation*}

$$
\frac{\partial \mathcal{L}}{\partial w_0} = 2w_0 + 2w_1 \frac{1}{N} \left( \sum_{n=1}^{N} x_n \right) -
\frac{2}{N} \left( \sum_{n=1}^{N} t_n \right)
$$

We obtain
\begin{align}
w_{1} & = \frac{\overline{xt} - \overline{x}\overline{t}}{\overline{x^2} - \overline{x}^2} \\
w_{0} & = \overline{t} - w_{1} \overline{x}
\end{align}
where symbols with overline denotes average value.

Example

Olympic 100m data

\begin{table}[H]
\begin{center}
\begin{tabular}{|c|c|}
\hline
Year & Seconds \\
\hline
1896 & 12.00 \\
1900 & 11.00 \\
1904 & 11.00 \\
...  & ...   \\
2008 & 9.69  \\
\hline
\end{tabular}
\end{center}
\end{table}


\begin{juliacode}
# Load data
data = readdlm("../../DATA/olympic100m.txt", ',')
x = data[:,1]
t = data[:,2]
xbar = mean(x)
tbar = mean(t)
w_1 = (mean(x.*t) - xbar*tbar)/(mean(x.^2) - xbar^2)
w_0 = tbar- w_1*xbar
\end{juliacode}

Result for the weights:
\begin{textcode}
w_0 = 36.41645590250286
w_1 = -0.01333088571096
\end{textcode}

\begin{figure}[H]
\begin{center}
\includegraphics[scale=0.75]{codes/IMG_linreg_simple.pdf}
\end{center}
\end{figure}


\subsection{Using matrix and vector notation}

We will rewrite our previous problem in matrix and vector notation. This will
give us more flexibility and enable us to generalize to more complex situations.
We start by defining inputs and model parameters as vectors.
\begin{equation*}
\mathbf{x}_{n} = \begin{bmatrix}
1 \\
x_{n}
\end{bmatrix}
,\,\,\,%
\mathbf{w} = \begin{bmatrix}
w_{0} \\
w_{1}
\end{bmatrix}
\end{equation*}

Using this definition, we can write our previous linear model in Eq.
\eqref{eq:model_linear_01} as:
\begin{equation}
f(x_n; w_0, w_1) = \mathbf{w}^{\mathsf{T}} \mathbf{x}_{n}
\label{eq:model_linear_02}
\end{equation}

The expression for loss function, Eq. \eqref{eq:loss_function_01}, becomes
\begin{equation}
\mathcal{L} = \frac{1}{N} \sum_{n=1}^{N} \left( t_{n} - \mathbf{w}^{\mathsf{T}}
\mathbf{x}_{n} \right)^2
\label{eq:loss_function_02}
\end{equation}
%
We now arrange several input vector into a matrix:
%
\begin{equation*}
\mathbf{X} = \begin{bmatrix}
\mathbf{x}^{\mathsf{T}}_{1} \\
\mathbf{x}^{\mathsf{T}}_{2} \\
\vdots \\
\mathbf{x}^{\mathsf{T}}_{N}
\end{bmatrix} =
\begin{bmatrix}
1 & x_{1} \\
1 & x_{2} \\
\vdots & \vdots \\
1 & x_{N} \\
\end{bmatrix}
\end{equation*}

As with inputs, we now define target vectors as
\begin{equation}
\mathbf{t} = \begin{bmatrix}
t_1 \\
t_2 \\
\vdots \\
t_N
\end{bmatrix}
\end{equation}
%
With this definition we can write the loss function as
\begin{equation}
\mathcal{L} = \frac{1}{N} \left( \mathbf{t} - \mathbf{Xw} \right)^{\mathsf{T}}
\left( \mathbf{t} - \mathbf{Xw} \right)
\end{equation}



To find the best value of $\mathbf{w}$ we can follow similar procedure that we have used
in the previous part. We need to find the solution of
$\dfrac{\partial \mathcal{L}}{\partial \mathbf{w}} = 0$

\begin{align}
\mathcal{L} & = \frac{1}{N} \left(
\mathbf{t}^{\mathsf{T}} \mathbf{t} +
\left(\mathbf{Xw}\right)^{\mathsf{T}} \mathbf{Xw} -
\mathbf{t}\mathbf{Xw} -
\left(\mathbf{Xw}\right)^{\mathsf{T}} \mathbf{t}
\right) \\
& = \frac{1}{N} \left(
\mathbf{w}^{\mathsf{T}} \mathbf{X}^{\mathsf{T}} \mathbf{X} \mathbf{w} -
2 \mathbf{w}^{\mathsf{T}} \mathbf{X}^{\mathsf{T}}\mathbf{t} +
\mathbf{t}^{\mathsf{T}} \mathbf{t}
\right)
\end{align}
Equating these to zeros we have
$$
\frac{\partial \mathcal{L}}{\partial \mathbf{w}} =
\frac{2}{N} \left( \mathbf{X}^{\mathsf{T}} \mathbf{Xw} - \mathbf{X}^{\mathsf{T}}\mathbf{t} \right) = 0
$$
So we have
\begin{equation}
\mathbf{X}^{\mathsf{T}} \mathbf{Xw} = \mathbf{X}^{\mathsf{T}} \mathbf{t}
\end{equation}
or
\begin{equation}
\highlighteq{
\mathbf{w} = \left(\mathbf{X}^{\mathsf{T}}\mathbf{X} \right)^{-1} \mathbf{X}^{\mathsf{T}} \mathbf{t}
}
\label{eq:w_vektor}
\end{equation}

\begin{juliacode}
data = readdlm("../../DATA/olympic100m.txt", ',')
x = data[:,1]
t = data[:,2]
Ndata = size(data,1)
# Build X matrix
X = zeros(Ndata,2)
for i in 1:Ndata
    X[i,1] = 1.0
    X[i,2] = x[i]
end
# Calculate w
w = inv(X' * X) * X' * t
println("w = ", w)
\end{juliacode}

Result:
\begin{textcode}
w = [36.416455902505334, -0.013330885710962845]
\end{textcode}

Tugas

Implementasikan program Python pada dataset yang sama dengan tugas sebelumnya
namun menggunakan notasi matriks-vektor sesuai Pers. \ref{eq:w_vektor}

\subsection{Generalization to more complex models}

We can use more complex models than \eqref{eq:model_linear_02}. For example
the quadratic equation:
\begin{equation}
f(x; w_{0}, w_{1}, w_{2}) = w_{0} +w_{1}x + w_{2}x^{2}
\label{eq:quadratic_eq}
\end{equation}
Note that the model is still linear in parameter, so this is still a linear model.
Using matrix and vector notation, we can fit the data to this equation simply by
adding one column to the matrix $\mathbf{X}$:
\begin{equation}
\mathbf{X} = \begin{bmatrix}
\mathbf{x}^{\mathsf{T}}_{1} \\
\mathbf{x}^{\mathsf{T}}_{2} \\
\vdots \\
\mathbf{x}^{\mathsf{T}}_{N}
\end{bmatrix} =
\begin{bmatrix}
1 & x_{1} & x_{1}^{2} \\
1 & x_{2} & x_{1}^{2} \\
\vdots & & \vdots \vdots \\
1 & x_{N} & x_{N}^{2} \\
\end{bmatrix}
\end{equation}


\subsection{Regularization}

We can define a measure of complexity of out linear model by
\begin{equation}
\sum_{i} w_{i}^{2}\,\,\text{atau }\mathbf{w}^{\mathsf{T}}\mathbf{w}
\end{equation}

As opposed to minimizing loss function \eqref{eq:loss_function_02}, we can minimize
a regularized loss function by adding penalty for overcomplexity:
\begin{equation}
\mathcal{L}' = \mathcal{L} + \lambda \mathbf{w}^{\mathsf{T}} \mathbf{w}
\end{equation}
where the arameter $\lambda$ control the trade off between model accuracy and
model complexity.
To find the best parameter, we can proceed by using similar procedure as before.
The regularized loss function is written as
\begin{equation}
\mathcal{L}' = \frac{1}{N} \mathbf{w}^{\mathsf{T}} \mathbf{X}^{\mathsf{T}} \mathbf{X} \mathbf{w}
- \frac{2}{N} \mathbf{w}^{\mathsf{T}} \mathbf{X}^{\mathsf{T}} \mathbf{t}
+ \frac{1}{N} \mathbf{t}^{\mathsf{T}} \mathbf{t}
+ \lambda \mathbf{w}^{\mathsf{T}} \mathbf{w}
\end{equation}
First derivative of the loss function with respect to model paramater $\mathbf{w}$ is
\begin{equation}
\frac{\partial \mathcal{L}'}{\partial \mathbf{w}} =
\frac{2}{N} \mathbf{X}^{\mathsf{T}} \mathbf{X} \mathbf{w}
- \frac{2}{N} \mathbf{X}^{\mathsf{T}} \mathbf{t} + 2\lambda\mathbf{w}
\end{equation}
Equating this to zero:
\begin{equation}
( \mathbf{X}^{\mathsf{T}} \mathbf{X} + N \lambda \mathbf{I} ) \mathbf{w} = \mathbf{X}^{\mathsf{T}} \mathbf{t}
\end{equation}
we obtain
\begin{equation}
\hat{\mathbf{w}} =
( \mathbf{X}^{\mathsf{T}} \mathbf{X} + N \lambda \mathbf{I} )^{-1}
\mathbf{X}^{\mathsf{T}} \mathbf{t}
\end{equation}



\bibliographystyle{unsrt}
\bibliography{BIBLIO}

\end{document}
