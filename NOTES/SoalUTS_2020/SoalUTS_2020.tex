\documentclass[a4paper,11pt]{article} % screen setting

\usepackage[a4paper]{geometry}
%\geometry{verbose,tmargin=1.5cm,bmargin=1.5cm,lmargin=1.5cm,rmargin=7.5cm}

\setlength{\parskip}{\smallskipamount}
\setlength{\parindent}{0pt}

%\usepackage{cmbright}
%\renewcommand{\familydefault}{\sfdefault}

%\usepackage{fontspec}
\usepackage[libertine]{newtxmath}
\usepackage[no-math]{fontspec}
\setmainfont{Linux Libertine O}
%\setmonofont{DejaVu Sans Mono}
\setmonofont{JuliaMono-Regular}


\usepackage{hyperref}
\usepackage{url}
\usepackage{xcolor}

\usepackage{amsmath}
\usepackage{amssymb}

\usepackage{graphicx}
\usepackage{float}

\usepackage{minted}

\newminted{julia}{breaklines,fontsize=\footnotesize}
\newminted{python}{breaklines,fontsize=\footnotesize}

\newminted{bash}{breaklines,fontsize=\footnotesize}
\newminted{text}{breaklines,fontsize=\footnotesize}

\newcommand{\txtinline}[1]{\mintinline[breaklines,fontsize=\footnotesize]{text}{#1}}
\newcommand{\jlinline}[1]{\mintinline[breaklines,fontsize=\footnotesize]{julia}{#1}}
\newcommand{\pyinline}[1]{\mintinline[breaklines,fontsize=\footnotesize]{python}{#1}}

\newmintedfile[juliafile]{julia}{breaklines,fontsize=\footnotesize}
\newmintedfile[pythonfile]{python}{breaklines,fontsize=\footnotesize}

\definecolor{mintedbg}{rgb}{0.90,0.90,0.90}
\usepackage{mdframed}
\BeforeBeginEnvironment{minted}{
    \begin{mdframed}[backgroundcolor=mintedbg,%
        topline=false,bottomline=false,%
        leftline=false,rightline=false]
}
\AfterEndEnvironment{minted}{\end{mdframed}}


\usepackage{setspace}

\onehalfspacing

\usepackage{appendix}


\newcommand{\highlighteq}[1]{\colorbox{blue!25}{$\displaystyle#1$}}
\newcommand{\highlight}[1]{\colorbox{red!25}{#1}}


\begin{document}


\title{Ujian Tengah Semester \\
TF4063 Sains Data dan Rekayasa}
\author{}
\date{14 Oktober 2020}
\maketitle

\begin{enumerate}
%
\item Jelaskan mengenai tiga cara pendekatan berikut untuk menentukan parameter model
pada mesin pembelajar:
  \begin{itemize}
  \item \textit{loss function minimization}
  \item \textit{maximum likelihood}
  \item \textit{Bayesian approach}
  \end{itemize}
%
\item (Kode Python) Buatlah data sintetik dengan menggunakan polinom: $f(x)=5x^3 - x^2 + x$ dan
diberikan noise Gaussian dengan $\mu=0$ dan $\sigma^2=300$. Ambil 100 nilai $x$
dari distribusi acak seragam dalam rentang $-5 < x < 5$. Gunakan hanya data dengan
$x < 0$ dan $x > 2$, artinya data yang berada pada $0 \geq x \geq 2$ akan dibuang.
Gunakan pendekatan \textit{maximum likelihood} untuk melakukan pemodelan pada data
ini dengan menggunakan polinom dengan orde 1 sampai dengan 8. Plot hasil yang Anda
peroleh beserta error bar yang terkait. Lengkapi jawaban Anda dengan kode Python
dan persamaan terkait yang digunakan (tidak perlu penurunan). Hasil yang Anda dapatkan
akan mereproduksi Gambar 2.17 pada buku Girolami-Rogers.
%
\item Jelaskan mengenai empat distribusi probabilitas yang terlibat
pada aturan Bayes (dalam kaitannya
dengan penentuan parameter model)
\begin{align}
p(\mathbf{w} | \mathbf{t}) = 
\frac{p(\mathbf{t}|\mathbf{w}) p(\mathbf{w})}{p(\mathbf{t})}
\end{align}
%
\item Jelaskan secara singkat tiga cara (aproksimasi) yang dapat dilakukan
untuk menentukan parameter model dengan pendekatan Bayesian jika distribusi
yang terlibat bukan konjugat.
%
\item Apakah yang dimaksud dengan regularisasi model? 
\end{enumerate}

\end{document}
