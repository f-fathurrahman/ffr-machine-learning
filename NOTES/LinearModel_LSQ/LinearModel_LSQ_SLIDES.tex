\documentclass[english,10pt,aspectratio=169,fleqn]{beamer}

\usepackage{amsmath} % load this before unicode-math
\usepackage{amssymb}
\usepackage{mathabx}
%\usepackage{unicode-math}

\usepackage{fontspec}
\setmonofont{DejaVu Sans Mono}
%\setmathfont{STIXMath}
%\setmathfont{TeX Gyre Termes Math}

\usefonttheme[onlymath]{serif}

\setlength{\parskip}{\smallskipamount}
\setlength{\parindent}{0pt}

%\setbeamersize{text margin left=5pt, text margin right=5pt}

\usepackage{amsmath}
\usepackage{amssymb}
\usepackage{braket}

\usepackage{minted}
\newminted{julia}{breaklines,fontsize=\scriptsize,texcomments=true}
\newminted{python}{breaklines,fontsize=\scriptsize,texcomments=true}
\newminted{bash}{breaklines,fontsize=\scriptsize,texcomments=true}
\newminted{text}{breaklines,fontsize=\scriptsize,texcomments=true}

\newcommand{\txtinline}[1]{\mintinline[fontsize=\scriptsize]{text}{#1}}
\newcommand{\jlinline}[1]{\mintinline[fontsize=\scriptsize]{julia}{#1}}

\definecolor{mintedbg}{rgb}{0.95,0.95,0.95}
\usepackage{mdframed}

%\BeforeBeginEnvironment{minted}{\begin{mdframed}[backgroundcolor=mintedbg]}
%\AfterEndEnvironment{minted}{\end{mdframed}}

\setcounter{secnumdepth}{3}
\setcounter{tocdepth}{3}

\makeatletter

 \newcommand\makebeamertitle{\frame{\maketitle}}%
 % (ERT) argument for the TOC
 \AtBeginDocument{%
   \let\origtableofcontents=\tableofcontents
   \def\tableofcontents{\@ifnextchar[{\origtableofcontents}{\gobbletableofcontents}}
   \def\gobbletableofcontents#1{\origtableofcontents}
 }

\makeatother

\usepackage{babel}

\newcommand{\highlighteq}[1]{\colorbox{blue!25}{$\displaystyle#1$}}
\newcommand{\highlight}[1]{\colorbox{red!25}{#1}}

\begin{document}

\title{Linear Model: Least Square Approach}
\subtitle{TF4063}
%\author{Tim Dosen Pengampu}
\author{Fadjar Fathurrahman}
\institute{
Program Studi Teknik Fisika\\
Institut Teknologi Bandung
}
\date{}


\frame{\titlepage}

\begin{frame} %------------------------

The material in this note is based on Rogers2017.

\end{frame} %--------------------------


\begin{frame} % -----------------------
\frametitle{Example dataset: {\tt olympic100m}}
\begin{columns}
  \begin{column}{0.4\textwidth}
  Let's start by studying a with a simple dataset.
  \end{column}
  \begin{column}{0.6\textwidth}
  {\centering
  \includegraphics[width=\textwidth]{codes_py/IMG_data_olympic100m.pdf}
  \par}
  \end{column}
\end{columns}
\end{frame} % -------------------------


\begin{frame} % -----------------------
\frametitle{Simple Linear Model}

\begin{columns}
  %
  \begin{column}{0.5\textwidth}
  Linear model:
  \begin{equation*}
  t = f(x; w_0, w_1) = w_0 + w_1 x
  \label{eq:model_linear_01}
  \end{equation*}
  \end{column}
  %
  \begin{column}{0.5\textwidth}
  Loss function:
  \begin{equation*}
  \mathcal{L} =
  \frac{1}{N} \sum_{n=1}^{N} \left( t_n - f(x_n; w_0, w_1) \right)^2
  \label{eq:loss_function_01}
  \end{equation*}
  \end{column}
  %
\end{columns}

\end{frame} % -----------------------------------------------------------------


\begin{frame} % ---------------------------------------------------------------
\frametitle{Finding the parameters}

Find the parameters $(w_{0},w_{1})$ by using minimization procedures:
\begin{equation*}
\arg\min_{w_{0},w_{1}} \frac{1}{N} \sum_{n=1}^{N} \mathcal{L}_{n}
\end{equation*}

For our particular case of Eq. \eqref{eq:loss_function_01}, we can found this analytically,
i.e. calculating the first derivatives of $\mathcal{L}$ with respect to $w_0$ and $w_1$, equating
them to zero, and solve the resulting equations for $w_0$ and $w_1$.

For more general cases, we can use various numerical optimization procedures such as
gradient descent methods.

\end{frame} % -----------------------------------------------------------------


\begin{frame} % ----------------------------
\frametitle{Finding the parameters}

We begin by writing our loss function as:
\begin{align*}
\mathcal{L} & = \frac{1}{N} \sum_{n=1}^{N} \left( t_n - (w_0 + w_1 x_{n}) \right)^2 \\
& = \frac{1}{N} \sum_{n=1}^{N} \left( w_1^2 x_n^2 + 2w_{1}x_{n}(w_0 - t_n) + w_0^2 - 2w_0 t_n + t_n^2 \right)
\end{align*}

Now we find the first derivatives of $\mathcal{L}$ with respect to
$w_0$, $w_1$ and equating them to zero.
\begin{align*}
\frac{\partial\mathcal{L}}{\partial w_1} & = 2w_1 \frac{1}{N} \left( \sum_{n=1}^{N} x_n^2 \right) +
\frac{2}{N} \left( \sum_{n=1}^{N} x_{n} (w_0 - t_n) \right) = 0 \\
\frac{\partial \mathcal{L}}{\partial w_0} & = 2w_0 + 2w_1 \frac{1}{N} \left( \sum_{n=1}^{N} x_n \right) -
\frac{2}{N} \left( \sum_{n=1}^{N} t_n \right) = 0
\end{align*}

\end{frame} % ---------------------------------------


\begin{frame} % -------------------------------------
\frametitle{Finding the parameters}

We obtain
\begin{align*}
\begin{split}
w_{1} & = \frac{\overline{xt} - \overline{x}\overline{t}}{\overline{x^2} - \overline{x}^2} \\
w_{0} & = \overline{t} - w_{1} \overline{x}
\end{split}
\label{eq:w0_w1_simple}
\end{align*}
where symbols with overline denotes their average value, for examples
\begin{align*}
\overline{x} & = \frac{1}{N} \sum_{n=1}^{N} x_{n} \\
\overline{t} & = \frac{1}{N} \sum_{n=1}^{N} t_{n}
\end{align*}

\end{frame} % ---------------------------------


\begin{frame} % -------------------------------
\frametitle{Application to {\tt olympic100m}}

XXX

\end{frame} % ---------------------------------


\begin{frame} % -------------------------------
\frametitle{Linear model: matrix-vector notation}

\begin{columns}

\begin{column}{0.5\textwidth}
Input and parameter vectors:
\begin{equation*}
\mathbf{x}_{n} \equiv \begin{bmatrix}
1 \\
x_{n}
\end{bmatrix}
,\,\,\,%
\mathbf{w} \equiv \begin{bmatrix}
w_{0} \\
w_{1}
\end{bmatrix}
\end{equation*}
%
Linear model:
\begin{equation*}
f(x_n; w_0, w_1) = \mathbf{w}^{\mathsf{T}} \mathbf{x}_{n}
\label{eq:model_linear_02}
\end{equation*}
%
Loss function:
\begin{equation}
\mathcal{L} = \frac{1}{N} \sum_{n=1}^{N} \left( t_{n} - \mathbf{w}^{\mathsf{T}}
\mathbf{x}_{n} \right)^2
\label{eq:loss_function_02}
\end{equation}
\end{column}
\begin{column}{0.5\textwidth}
Design matrix:
\begin{equation*}
\mathbf{X} = \begin{bmatrix}
\mathbf{x}^{\mathsf{T}}_{1} \\
\mathbf{x}^{\mathsf{T}}_{2} \\
\vdots \\
\mathbf{x}^{\mathsf{T}}_{N}
\end{bmatrix} =
\begin{bmatrix}
1 & x_{1} \\
1 & x_{2} \\
\vdots & \vdots \\
1 & x_{N} \\
\end{bmatrix}
\end{equation*}
  
\begin{equation*}
\mathbf{t} = \begin{bmatrix}
t_1 \\
t_2 \\
\vdots \\
t_N
\end{bmatrix}
\end{equation*}
\end{column}
\end{columns}

\end{frame}


\begin{frame} % -----------------------------
\frametitle{Finding the parameters}

\begin{columns}
  \begin{column}{0.5\textwidth}
  \begin{equation*}
  \mathcal{L} = \frac{1}{N} \left( \mathbf{t} - \mathbf{Xw} \right)^{\mathsf{T}}
  \left( \mathbf{t} - \mathbf{Xw} \right)
  \end{equation*}
  %
  \begin{equation*}
  \dfrac{\partial \mathcal{L}}{\partial \mathbf{w}} = 0
  \end{equation*}
  %
  \begin{align*}
  \mathcal{L} & = \frac{1}{N} \left(
  \mathbf{t}^{\mathsf{T}} \mathbf{t} +
  \left(\mathbf{Xw}\right)^{\mathsf{T}} \mathbf{Xw} -
  \mathbf{t}\mathbf{Xw} -
  \left(\mathbf{Xw}\right)^{\mathsf{T}} \mathbf{t}
  \right) \\
  & = \frac{1}{N} \left(
  \mathbf{w}^{\mathsf{T}} \mathbf{X}^{\mathsf{T}} \mathbf{X} \mathbf{w} -
  2 \mathbf{w}^{\mathsf{T}} \mathbf{X}^{\mathsf{T}}\mathbf{t} +
  \mathbf{t}^{\mathsf{T}} \mathbf{t}
  \right)
  \end{align*}
  \end{column}
  %
  \begin{column}{0.5\textwidth}
  \begin{equation*}
  \frac{\partial \mathcal{L}}{\partial \mathbf{w}} =
  \frac{2}{N} \left( \mathbf{X}^{\mathsf{T}} \mathbf{Xw} - \mathbf{X}^{\mathsf{T}}\mathbf{t} \right) = 0
  \end{equation*}
  %
  \begin{equation*}
  \mathbf{X}^{\mathsf{T}} \mathbf{Xw} = \mathbf{X}^{\mathsf{T}} \mathbf{t}
  \end{equation*}
  %
  \begin{equation*}
  \highlighteq{
  \mathbf{w} = \left(\mathbf{X}^{\mathsf{T}}\mathbf{X} \right)^{-1} \mathbf{X}^{\mathsf{T}} \mathbf{t}
  }
  \label{eq:w_vektor}
  \end{equation*}
  \end{column}
\end{columns}
\end{frame} % ---------------------------------


\begin{frame}[fragile] % -------------------------------
\frametitle{Application to {\tt olympic100m} (using matrix-vector notation)}

\begin{pythoncode}
def hello():
  pass
\end{pythoncode}

\end{frame}


\begin{frame} % -------------------------------
\frametitle{Example slide}
\begin{columns}
  \begin{column}{0.5\textwidth}
  Column 1
  \end{column}
  \begin{column}{0.5\textwidth}
  Column 2
  \end{column}
\end{columns}
\end{frame}



\end{document}

