\documentclass[english,10pt,aspectratio=169,fleqn]{beamer}

\usepackage{amsmath} % load this before unicode-math
\usepackage{amssymb}
\usepackage{mathabx}
%\usepackage{unicode-math}

\usepackage{fontspec}
\setmonofont{DejaVu Sans Mono}
%\setmathfont{STIXMath}
%\setmathfont{TeX Gyre Termes Math}

\usefonttheme[onlymath]{serif}

\setlength{\parskip}{\smallskipamount}
\setlength{\parindent}{0pt}

%\setbeamersize{text margin left=5pt, text margin right=5pt}

\usepackage{amsmath}
\usepackage{amssymb}
\usepackage{braket}

\usepackage{minted}
\newminted{julia}{breaklines,fontsize=\scriptsize,texcomments=true}
\newminted{python}{breaklines,fontsize=\scriptsize,texcomments=true}
\newminted{bash}{breaklines,fontsize=\scriptsize,texcomments=true}
\newminted{text}{breaklines,fontsize=\scriptsize,texcomments=true}

\newcommand{\txtinline}[1]{\mintinline[fontsize=\scriptsize]{text}{#1}}
\newcommand{\jlinline}[1]{\mintinline[fontsize=\scriptsize]{julia}{#1}}

\definecolor{mintedbg}{rgb}{0.95,0.95,0.95}
\usepackage{mdframed}

%\BeforeBeginEnvironment{minted}{\begin{mdframed}[backgroundcolor=mintedbg]}
%\AfterEndEnvironment{minted}{\end{mdframed}}

\setcounter{secnumdepth}{3}
\setcounter{tocdepth}{3}

\makeatletter

 \newcommand\makebeamertitle{\frame{\maketitle}}%
 % (ERT) argument for the TOC
 \AtBeginDocument{%
   \let\origtableofcontents=\tableofcontents
   \def\tableofcontents{\@ifnextchar[{\origtableofcontents}{\gobbletableofcontents}}
   \def\gobbletableofcontents#1{\origtableofcontents}
 }

\makeatother

\usepackage{babel}

\begin{document}


\title{Singular Value Decomposition}
\subtitle{TF4063}
\author{Fadjar Fathurrahman}
\institute{
Program Studi Teknik Fisika\\
Institut Teknologi Bandung
}
\date{}


\frame{\titlepage}


\begin{frame} % ---------------------------------------------------------------
\frametitle{SVD}

A large data set $\mathbf{X} \in \mathbb{C}^{n \times m}$:
\begin{equation*}
\mathbf{X} = \begin{bmatrix}
\mathbf{x}_{1} & \mathbf{x}_{2} & \cdots & \mathbf{x}_{m} \\
\end{bmatrix}
\end{equation*}

SVD is a unique matrix decomposition that is defined by:
\begin{equation*}
\mathbf{X} = \mathbf{U}\boldsymbol{\Sigma}\mathbf{V}^{*}
\end{equation*}
where $\mathbf{U} \in \mathbb{C}^{n \times n}$ and
$\mathbf{V} \in \mathbb{C}^{m \times m}$ are unitary matrices with orthonormal
columns, and $\boldsymbol{\Sigma} \in \mathbb{R}^{n \times m}$ is a matrix with
real, nonnegative entries on the diagonal and zeros off the diagonal.

\end{frame} % -----------------------------------------------------------------


\begin{frame}[plain]

\includegraphics[scale=0.8]{images_priv/Brunton_Fig_1_1.pdf}

\end{frame}


\begin{frame}
\frametitle{Matrix approximation}

SVD provides an optimal low-rank approximation to a matrix $\mathbf{X}$.

The rank of a matrix is defined as (a) the maximum number of
linearly independent column vectors in the matrix or
(b) the maximum number of linearly independent row vectors in the matrix

\begin{equation*}
\tilde{\mathbf{X}} = \sum_{k=1}^{r} \sigma_{k} \mathbf{u}_{k} \mathbf{v}^{*}_{k}
= \sigma_{1} \mathbf{u}_{1} \mathbf{v}^{*}_{1} +
\sigma_{2} \mathbf{u}_{2} \mathbf{v}^{*}_{2} + \cdots
\sigma_{r} \mathbf{u}_{r} \mathbf{v}_{r}
\end{equation*}

\end{frame}


\begin{frame}[plain]

\includegraphics[scale=0.8]{images_priv/Brunton_Fig_1_2.pdf}

\end{frame}


\begin{frame}
\frametitle{Example: Approximating an image}

Gray scale image

\end{frame}


\begin{frame}
\frametitle{}

Linear system of equations:
\begin{equation*}
\mathbf{A}\mathbf{x} = \mathbf{b}
\end{equation*}

\end{frame}



\end{document}

