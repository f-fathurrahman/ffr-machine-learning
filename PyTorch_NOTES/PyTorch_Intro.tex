\documentclass[a4paper,11pt]{article} % screen setting

\usepackage[a4paper]{geometry}
\geometry{verbose,tmargin=1.5cm,bmargin=1.5cm,lmargin=1.5cm,rmargin=1.5cm}

\setlength{\parskip}{\smallskipamount}
\setlength{\parindent}{0pt}

%\usepackage{cmbright}
%\renewcommand{\familydefault}{\sfdefault}

%\usepackage{fontspec}
\usepackage[libertine]{newtxmath}
\usepackage[no-math]{fontspec}
\setmainfont{Linux Libertine O}
%\setmonofont{DejaVu Sans Mono}
\setmonofont{JuliaMono-Regular}


\usepackage{hyperref}
\usepackage{url}
\usepackage{xcolor}

\usepackage{amsmath}
\usepackage{amssymb}

\usepackage{graphicx}
\usepackage{float}

\usepackage{minted}

\newminted{julia}{breaklines,fontsize=\footnotesize}
\newminted{python}{breaklines,fontsize=\footnotesize}

\newminted{bash}{breaklines,fontsize=\footnotesize}
\newminted{text}{breaklines,fontsize=\footnotesize}

\newcommand{\txtinline}[1]{\mintinline[breaklines,fontsize=\footnotesize]{text}{#1}}
\newcommand{\jlinline}[1]{\mintinline[breaklines,fontsize=\footnotesize]{julia}{#1}}
\newcommand{\pyinline}[1]{\mintinline[breaklines,fontsize=\footnotesize]{python}{#1}}

\newmintedfile[juliafile]{julia}{breaklines,fontsize=\footnotesize}
\newmintedfile[pythonfile]{python}{breaklines,fontsize=\footnotesize}

\definecolor{mintedbg}{rgb}{0.90,0.90,0.90}
\usepackage{mdframed}
\BeforeBeginEnvironment{minted}{
    \begin{mdframed}[backgroundcolor=mintedbg,%
        topline=false,bottomline=false,%
        leftline=false,rightline=false]
}
\AfterEndEnvironment{minted}{\end{mdframed}}


\usepackage{setspace}

\onehalfspacing

\usepackage{appendix}


\newcommand{\highlighteq}[1]{\colorbox{blue!25}{$\displaystyle#1$}}
\newcommand{\highlight}[1]{\colorbox{red!25}{#1}}



\begin{document}


\title{PyTorch Tutorial\\
TF4063}
\author{Fadjar Fathurrahman}
\date{}
\maketitle

\section{Linear regression}

\begin{pythoncode}
import torch
import torch.nn as nn
import numpy as np
import matplotlib.pyplot as plt
    
RANDOM_SEED = 1234
np.random.seed(RANDOM_SEED)
torch.manual_seed(RANDOM_SEED)
    
def create_dataset(Ndata):
    w_true = [0.25, 1.1]
    noise_var = 0.01
    x_train = -5.0 + 10*np.random.rand(Ndata)
    y_noise = np.sqrt(noise_var)*np.random.randn(Ndata)
    y_train = w_true[1]*x_train + w_true[0] + y_noise
    return x_train, y_train

Ndata = 20
x_train, y_train = create_dataset(Ndata)
plt.clf()
plt.plot(x_train, y_train, linewidth=0, marker="o")
plt.savefig("IMG_data.pdf")

# Convert data to tensor, we need to reshape it first
x_train_ = np.reshape(x_train, (Ndata,1))
y_train_ = np.reshape(y_train, (Ndata,1))
# Default floating point number in PyTorch is float32
inputs = torch.tensor(x_train_, dtype=torch.float32)
targets = torch.tensor(y_train_, dtype=torch.float32)

# Hyper-parameters
input_size = 1 # only one feature
output_size = 1 # 
num_epochs = 200
learning_rate = 0.1

# Linear regression model
model = nn.Linear(input_size, output_size)
    
# Loss and optimizer
criterion = nn.MSELoss()
optimizer = torch.optim.SGD(model.parameters(), lr=learning_rate)

# Train the model
for epoch in range(num_epochs):
    # Forward pass
    outputs = model(inputs)
    loss = criterion(outputs, targets)
    # Backward and optimize
    optimizer.zero_grad()
    loss.backward()
    optimizer.step()        
    if (epoch+1) % 5 == 0:
        print ('Epoch [{}/{}], Loss: {:.4f}'.format(epoch+1, num_epochs, loss.item()))

print("Parameters")
for name, param in model.named_parameters():
    print(name, " ", param.data)
    
# Plot the graph
predicted = model(inputs).detach().numpy()
plt.plot(x_train, y_train, 'ro', label='Original data')
plt.plot(x_train, predicted, label='Fitted line')
plt.legend()
plt.show()
    
# Save the model checkpoint
torch.save(model.state_dict(), 'model.ckpt')
\end{pythoncode}


\section{Logistic regression}


\begin{pythoncode}
import torch
import torch.nn as nn
import torchvision
import torchvision.transforms as transforms
import numpy as np

RANDOM_SEED = 1234
np.random.seed(RANDOM_SEED)
torch.manual_seed(RANDOM_SEED)

# Hyper-parameters 
input_size = 28 * 28    # 784
num_classes = 10
num_epochs = 5
batch_size = 100
learning_rate = 0.001

# MNIST dataset (images and labels)
train_dataset = torchvision.datasets.MNIST(
    root='../DATASET', 
    train=True, 
    transform=transforms.ToTensor(),
    download=False
)
    
test_dataset = torchvision.datasets.MNIST(
    root='../DATASET', 
    train=False, 
    transform=transforms.ToTensor()
)

# Data loader (input pipeline)
train_loader = torch.utils.data.DataLoader(
    dataset=train_dataset, 
    batch_size=batch_size, 
    shuffle=True
)

test_loader = torch.utils.data.DataLoader(
    dataset=test_dataset, 
    batch_size=batch_size, 
    shuffle=False
)
    
# Logistic regression model
model = nn.Linear(input_size, num_classes)
    
# Loss and optimizer
criterion = nn.CrossEntropyLoss()  
optimizer = torch.optim.SGD(model.parameters(), lr=learning_rate)  

# Train the model
total_step = len(train_loader)
for epoch in range(num_epochs):

    # Loop over all data, batch by batch
    for i, (images, labels) in enumerate(train_loader):
        # Reshape images to (batch_size, input_size)
        images = images.reshape(-1, input_size)

        # Forward pass
        outputs = model(images)
        loss = criterion(outputs, labels)
            
        # Backward and optimize
        optimizer.zero_grad()
        loss.backward()
        optimizer.step()
            
        if (i+1) % 100 == 0:
            print ('Epoch [{}/{}], Step [{}/{}], Loss: {:.4f}' 
                   .format(epoch+1, num_epochs, i+1, total_step, loss.item()))
    
# Test the model
# In test phase, we don't need to compute gradients (for memory efficiency)
with torch.no_grad():
    correct = 0
    total = 0
    for images, labels in test_loader:
        images = images.reshape(-1, input_size)
        outputs = model(images)
        _, predicted = torch.max(outputs.data, 1)
        total += int(labels.size(0))
        correct += int((predicted == labels).sum())
    print('Accuracy of the model on the 10000 test images: {} %'.format(100 * int(correct) / int(total)))
    
# Save the model checkpoint
torch.save(model.state_dict(), 'model.ckpt')    
\end{pythoncode}


\bibliographystyle{unsrt}
\bibliography{BIBLIO}

\end{document}
